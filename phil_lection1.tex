\documentclass[a4paper,12pt]{report} %размер бумаги устанавливаем А4, шрифт 12пунктов
\usepackage[T2A]{fontenc}
\usepackage[utf8]{inputenc} %включаем свою кодировку: koi8-r или utf8 в UNIX, cp1251 в Windows
\usepackage[english,russian]{babel} %используем русский и английский
                                %языки с переносами
\usepackage[pdftex,unicode]{hyperref}
\usepackage{amssymb,amsfonts,amsmath,mathtext,cite,enumerate,float} %подключаем нужные пакеты расширений

\usepackage{geometry} % Меняем поля страницы
\geometry{left=1cm} % левое поле
\geometry{right=1cm} % правое поле
\geometry{top=1cm} % верхнее поле
\geometry{bottom=2cm} % нижнее поле

\renewcommand{\theenumi}{\arabic{enumi}} % Меняем везде перечисления на цифра.цифра
\renewcommand{\labelenumi}{\arabic{enumi}} % Меняем везде перечисления на цифра.цифра
\renewcommand{\theenumii}{.\arabic{enumii}} % Меняем везде перечисления на цифра.цифра
\renewcommand{\labelenumii}{\arabic{enumi}.\arabic{enumii}.} % Меняем везде перечисления на цифра.цифра
\renewcommand{\theenumiii}{.\arabic{enumiii}} % Меняем везде перечисления на цифра.цифра
\renewcommand{\labelenumiii}{\arabic{enumi}.\arabic{enumii}.\arabic{enumiii}.}
% Меняем везде перечисления на цифра.цифра


\begin{document}
\begin{itemize}
\item \textbf{Эпоха возрождения}
\\ 
\textit{Основные черты: } Античность(искусство, внутренний мир человека, человеческое тело,
человеческий дом, пантеизм(Бог тождественен природе), языки(греческий,
латынь)), астрология,магия, алхимия.
\item Никколо Макиавели (1469 - 1527). Флоренция. \\
Взгляды:
\begin{itemize}
\item Политика: смотреть на жизнь реально. Люди ни хорошие, ни плохие,
  но больше плохие. Люди ценят силу. Вирту(добродетель) имел в разное
  время разное значение. У Макиавели вирту - это сила. Цель
  оправдывает средства для государя. Правитель должен убрать все, что
  было до него.
\item Бог это фортуна, фортуна это женщина, женщины любят силу, нужно
  брать ее в свои руки и управлять ей. Религия нужна. Она как цемент
  для государства.
\end{itemize}
\item Томас Мор (1478 - 1535). \\
Утопия - проект политического преобразования, за основу взяты идеи
Платона. Это всего лишь модель. Не претендует на реализацию в
жизни. Частная собственность вредит. Большая роль отводилась религии,
но в государстве провозглашена свобода вероисповедания. Рабочий день
составлял 6 часов. От каждого по способностям, каждому по
потребностям. Имело место рабство. Ручной труд.
\item Томмазо Компанелла (1568 - 1639). \\
Отсутствие частной собственности. Делал ставку на технический
прогресс, каждый житель должен обучаться ремеслу. Нет негодяев и
тунеядцев, все работают, каждый обучается военному делу. Люди
обмениваются домами каждые полгода. Человек должен трудиться 4 часа в
день, а в последствии еще меньше. Управляется государство верховным
жрецом (метафизиком). У него три жреца в подчинении: могущество(война и мир), мудрость (наука и
просвещение), любовь(устройство браков). Селекционирование людей. Если
люди полюбили друг друга, то они должны были спросить разрешения у
жреца. Если жрец против, то они могли дарить друг другу венки и писать
стихи. Вознаграждение определяется начальником. Религия существует. Но
странная. Большую роль играет астрология. Люди с ограниченными
возможностями тоже могут заниматься делами.\\
\textbf{Гносиология (познание), сенсуалисты(эмпирики)(в основе познания лежат чувства) и
рационалисты(первична мысль, разум человека). Метод индукции(от
частного к общему) - эмпирики, рационалисты - дедукция. Метод
монументальной пропаганды (плакаты).
}
Компанелла эмпирик, но считает, что разум важен. Потому что есть
галлюцинации, сны и т.д. Сущность нам недоступна, мы можем познавать
только явления.
\item Мартин Лютер \\
Вывесил 95 тезисов на дверях университета. Они очень быстро
распространились по всей Германии.\\
\begin{enumerate}
\item Церковь не является посредником между Богом и землей.
\item Священник: крещение, отпевание
\item Священник объясняет людям библию
\item Принцип спасения верой - человек спасается личной верой, а не
  выполнением церковных обрядов. Верующий человек не может совершать
  плохие дела.
\item Отменить все виды оплаты.
\item Закрыть монастыри - аскеза не нужна, нужна сдержанность.
\item Целебат(безбрачие) отменяется
\item Никаких индульгенций
\end{enumerate}
Протестанты Франции, северных стран, Германии, Испании, Италии,
Польши, Швейцарии быстро узнали эту доктрину. А в Швейцарии был
кальвенизм. Жан Кальвин. Принцип божественного предопределения: Бог уже
знает, кто попадет в ад, а кто в рай. Чтобы выяснить куда попадешь
нужно неустанно работать. Если ты богат, то попадешь в рай. Но при
этом все жили в аскезе. Никакой роскоши, все дома открыты,
обязательное посещение проповедей Кальвина. Так получилось, что
протестантские страны стали богаче остальных.\\
\\
\textbf{Лекция 2}\\
\begin{itemize}
\item Коперник, Николай (1473-1543)\\
Астрономия была его хобби, он был юристом(каноником - церковным
юристом), монахом. Так же увлекался медициной и лечил бедняков
бесплатно. Отразил нападение Тевтонского Ордена, провел денежную
реформу, боролся с эпидемиями, участвовал в дипломатических миссиях. \textbf{НИКТО ЕГО НЕ ЖЕГ НА КОСТРЕ.}\\
Написал работу ``Об обращении небесных сфер''. Церковь на эту работу
никак не отреагировала. Работа была признана еретической только
1616г.\\
Ближе к неоплатонизму, путь солнца, герметические учения. Бог --
геометр, создал мир в соответствии с математикой. ``Мир
сферичен''. \textit{Гелиоцентрическая система. Земля вращается вокруг Солнца и
вокруг своей оси}.
\item Бруно, Джордано (1548-1600)\\
При рождении получил имя Филипп. Когда стал монахом, ему дали имя в
честь реки Иордан. По слухам убил монаха в монастыре и ему пришлось
бежать. Путешествовал по городам Италии, побывал в Швейцарии. А в
Швейцарии были кальвинисты уже, он с ними подружился, а потом
рассорился и отправился во Францию. Познакомился с королем Генрихом
3. Отправился в Лондон, где познакомился с Оксфордскими
преподавателями, назвал их педантами и они обиделись. Назвали его
плагиатором. Вернулся в Париж и познакомился с последователями
Аристотеля(схоластами). Поссорился с ними и бежал в
Германию. Познакомился с лютеранами, вступил в их общину и был
изгнан. Вернулся в Италию, где его попросили заняться преподаванием
тренеровки памяти. Вспылил и оскорбил ученика, а тот написал на него
донос. Утверждал, что Иисус просто человек, исповедовал идею
реинкарнации, говорил, что небесные тела одухотворены и поклонялся
Солнцу, был пантеистом. \underline{Миров много}. Гелиоцентризм объяснялся
культом Солнца. Звезды -- это солнца.

Занимался тренировкой памяти и разработал доктрину для развития
памяти. В основе памяти лежат архетипы, с помощью которых можно
обрести космическую силу.
\item Тихо Браге (1546-1601)\\
Происходил из знатной семьи. Поступил в Копенгагенский университет в
12 лет. Учился на юриста. Его поразило затмение Солнца и побудило
заниматься астрономией. В 16 лет уехал учиться в Германию на 6
лет. Иногда заезжал в Данию. Видел соединение Юпитера с Сатурном, что
тоже его поразило. Обладая острым зрением, измерения проводил с
помощью одного лишь циркуля. Поссорился с другом, был вызван на дуэль,
где ему шпагой отсекли часть носа -- был вынужден носить серебряные
протезы. Из-за этого у него развились комплексы и он никогда не
женился. После смерти отца наладил стекольное производство(был
алхимиком и астрологом). И занимался астрономией. Наблюдал сверхновую
и написал целое произведение ``О новой звезде''. Вспыхнула она ровно
через месяц после Варфоломеевской ночи и начались толки о конце
света. Фридрих 2 заметил Тихо Браге и дал ему в пожизненное владение
остров. Там Браге построил обсерваторию и назвал ее небесный
замок. Достиг фантастической точности в своих наблюдениях. \underline{Он выяснил,
что небесных сфер нет}. Подсчитал все небесные тела и составил
глобус. Количество звезд оценил где-то тысячью. Думал, что Земля
неподвижна. В конце жизни встретился с Кеплером. Они поссорились и
Браге вскоре умер. Он хотел, чтобы Кеплер стал его приемником.
\item Кеплер, И. (1571-1630)\\
Сын служителя лютеранской церкви. Родился болезненным, переболел оспой
в детстве. Из-за болезней в детстве у него было слабое зрение. Его
мама подсыпала в напитки посетителям трактира разные травы и наблюдала
за их поведением. Поступил в университет, а хотел стать
священнослужителем. Но в 22 года он передумал и начал преподавать
математику и делал прогнозы различных событий(волнения крестьян,
осадки). Переписывался с Галилеем. Кеплер женился на богатой вдове,
император назначил его придворным математиком. Получал немалые деньги
от императора.\\
Занимался проблемой света. \underline{Изображение попадает на сетчатку в
перевернутом виде. О стереометрии, вычислял объем бочки методом,
близким к дифференциальному. Законы Кеплера.} Делал все это, как
маг. Мир описывается математическими законами. Ощущения дают
хаотическую массу информации, которую обрабатывает разум. К учению
Коперника относится крайне восторженно.
\item Галилей, Галилео (1564-1642)\\
С детства проявил склонность к конструированию. Интересовался
медициной, богословием. Любил пребывать в изуитском монастыре, но
монахом не стал. Пригласили преподавать медицину ординарным(деньги
платит университет)
преподавателем, но он стал преподавать математику
неординарным(деньги платили студенты)
преподавателем. Его интересовала теология, изучал круги ада
Данте. Уезжает преподавать в Голландию. Там пишет большую часть своих
работ. Там он познакомился с телескопом. Нашел патрона, который ему
покровительствовал. Увлекся идеями Коперника. Его защищал Папа, но
Гилилей написал произведение, высмеивающее церковь и Папа
обиделся. Была у него дочь - монахиня, он ездил к ней в
монастырь. Когда она умерла, он ``расклеился''. Его сожгли.\\
Устройство космоса по Аристотелю: подлунный и надлунный мир, небесные
сферы, Бог -- перводвигатель. Галилей \textit{сделал супер-телескоп x30} и сказал,
что \underline{нет никакого надлунного мира}. Исповедует идеи Коперника,
ставит Солнце в центр мира. \underline{Разработал идею
математического(мысленного) эксперимента, что бы было в идеальных
условиях. Теория двойственной истины. Наука и вера несовместимы. Бог
дал нам разум, чувства. А знания получаем посредством чувственного
опыта. А разум дан, чтобы знания удостоверить. Галактика -- скопление
звезд. Священное писание -- не трактат по астрономии, там не нужно
искать объяснения творения мира.} Священное писание не учит жить на
земле, а учит как душе попасть в рай. Экзегетика(толкование
библии). \textit{Писание правильно истолковать может только ученый.}
Не следует искать в писании ответы на вопросы, на которые человек
может ответить своим разумом. \textit{Наука нейтральна к миру ценностей, вера
некомпетентна в вопросах факта. Какова сила истины: вы пытаетесь ее
опровергнуть, но сами ваши нападки возвышают ее и придают ей большую ценность}.
\end{itemize}
\begin{enumerate}
\itemНаучная революция (от создания Коперником ``Обращение небесных сфер''
до ``Начала'' Ньютона). Создание Гелиоцентрической системы.
\itemУниверситеты(схоластика) уступили место академиям(научный метод:
  наблюдение, эксперимент).
\itemОживление контактов между учеными. Распространение знаний с
  помощью книгопечатания.
\itemУченые не были атеистами, идея Бога видоизменяется. Наука и
  христианство соседствуют с магией, астрологией, оккультными
  практиками.
\itemУченый -- новый облик, человек, который в основу кладет
  эксперимент. Все инструменты и эксперименты делали сами.
\itemСближаются ученые и ремесленники. Ученый -- человек деятельный.
\end{enumerate}
\item Что-то новенькое
\begin{itemize}
\itemБэкон Ф. (1561-1626)\\
Родился в богатой, знатной семье, в Англии. Отец был хранителем государственной
печати, мать переводила с латыни. Устроился на гос. службу. Начал
выступать против политики Елизаветы. Но позже Яков его приблизил к
себе и покровительствовал ему(Бэкону).\\
Считается родоначальником современной философии(\textbf{эмпирики}).\\
Аристотеля не уважал, вообще не очень относился к древним
философам. Бэкон говорит, что философия древних -- это детство, нужно
все переписывать. Так появились ``Новая Атлантида'' и ``Новый
органон''. Теперь во главу угла ставится метод. Бэкон: ``Знание --
сила''. В царство науки нужно входить чистым, как дитя.\\
Препятствия на пути к познанию:
\begin{enumerate}
\itemИдол рода(органы чувств)
\itemИдол пещеры(свои заморочки, привычки)
\itemИдол рынка(слухи)
\itemИдол театра(капание на мозги, типа рекламы)
\end{enumerate}
Предлагает отказываться от авторитетов. Истина -- дочь времени, а не
авторитета. Причины заблуждений: софистика, эмпирика(опыт
недостаточен), суеверия.\\
Атеизм -- тонкий лед, по которому один человек пройдет, а нация
провалится.\\
Индукция -- от частного к общему. Индукция это метод Бэкона. Таблица
присутствия, отсутствия и степеней.\\
Муравьи -- эмпирики, собирающие знания, пауки -- рационалисты,
развивающие идеи только из своей головы, пчелы -- ученые.\\
Плодоносные опыты -- дают возможность использовать результат в будущем, светоносные опыты -- помогают открыть скрытые
причины явления.\\
Классификация наук:\\
В основе -- способности человека:\\
Память, воображение, рассудок(философия: физика и метафизика).\\
Науки: прикладные(физика -- механика, метафизика(магия)) и
теоретические.\\
Стал жертвой опыта: проводил опыт с замороженной курицей, простудился
и умер.
\itemЛокк Д. (1632-1704)\\
Обучался в хорошей школе, несмотря на то, что был небогат. Отказался
от Аристотеля. Был хорошим врачом и служил у знатного лорда Эшли. Был
принят в Лондонское Королевское Общество. В конце жизни поселился в
замке знатной дамы.\\
Гносиология. Сенсуалист -- в основе познания лежат чувства. Сознание
ребенка -- чистая доска. Идея(из внешнего мира, рефлексивные(путем
размышлений)). Рефлексии вторичны -- возникают на основе чувственного
опыта. Идеи сложные: путем суммирования, сопоставления, выделение
общих черт через абстракцию. Каждому предмету присущи первичные
качества, которые реально существуют в телах(вес, форма), и
другие(цвет, вкус).\\
 Чувственное познание -- самое достоверное. Демонстративное
 познание(разум) -- умозаключения. Интуитивное познание -- озарение.\\
В государстве должно существовать разделение властей(исполнительная,
законодательная). Существуют естественные права. Свобода совести --
это свобода вероисповедания.\\
Локк говорит, как надо воспитывать детей. Надо воспитывать ребенка
целенаправленно, разностороннее обучение ему давать.
\itemГоббс Т. (1588-1679)\\
Окончил Оксфорд(поступил в 15 лет). Отлично знал латынь. Устроился
воспитателем к богатому джентльмену, путешествовал вместе с ним.\\
Гносиология. Гоббс был человеком с крупным телом. В трактате ``О
теле'' он пишет, что тела бывают естественные(материальные) и
искусственные(государство). С помощью законов физики можно описывать и
политику. Природа -- совокупность протяженных тел, различающихся
величиной, фигурой. Неотъемлемые свойства -- протяженность и
фигура. Остальные качества -- фантомные. Пространство -- это
воображаемый образ существующей вне нас вещи, а время -- это образ ее
движения. Теория языка, как знаковой системы. Истина -- понятие
абстрактное, о ней говорить бессмысленно, научное знание
относительно. Нужно изучать только реальные вещи. Гипотеза -> дедукция
должна сосуществовать с индукцией. Подчеркивал значение интуиции.\\
``Левиофан'' -- государство. \underline{Теория общественного
  договора}.\\
Деятельность человека невозможна без осознания им своей
свободы. Естественные права. Идеальная форма правления -- монархия.\\
Соперничество, недоверие и жажда славы.\\
Деизм(Теизм?) -- Бог создал мир и удалился. Религия должна быть
подчинена государству. Убрать мистику, астрологию, магию из
религии. Религия должна быть удобной для государства. Главное --
практическая выгода, он был прагматиком.
\itemБеркли, Джордж(1685-1753)\\
Семья среднего достатка. Стал англиканским священником после
колледжа. Стал профессором и преподавал языки. Двинул в
Америку. Познакомился со Свифтом. Вернулся в Лондон, стал деканом. ``О
пользе дегтярной настойки''.\\
\underline{Существовать значит быть воспринимаемым}. Беркли не нравился
материализм. Слова служат для обозначения не реальных предметов, а
наших идей. Идеи возникают на основе ощущений или рождаются в нашем
разуме, все это в результате восприятия мира. Первичных качеств нет --
все субъективно. Не отрицал существование мира, но о материи говорить
бессмысленно, потому что сам термин придумал человек. Реляционная
концепция пространства-времени. Отрицал абстрактные идеи. Он был
\underline{номиналистом}. Все предметы существуют изначально в
сознании Бога, а он развил их в сознании человека. 
\itemЮнг(ктоблять?), Дэвид (1711-1776)\\
``Наука о человеческой природе''. Книгу не заметили и у него начался
  духовный кризис. Он очень хотел стать известным и стать профессором
  в Оксфорде, куда его не пускали из-за плохого отношения к
  религии. Пытался подружиться с Руссо. Юнг помогал материально Руссо
  и Руссо подумал, что Юнг хочет его уничтожить. Написал трактат по
  истории и его заметили, начали считать крутым историком. Прежде чем
  начинать процесс познания, нужно понять природу человека, как мы
  познаем мир. Агностицизм -- мир познать невозможно. Мы не можем
  ничего говорить о мире с полной уверенностью, потому что все это --
  наше впечатление. Впечатления делятся на простые и сложные. Цвет,
  тепло, форма -- простые, цветок -- сложное. Восприятие формирует
  идеи, память дает возможность создавать идеи. Человек -- чистый
  лиц. Свобода. Номинализм. Страсти и аффекты. Страсти
  делятся на прямые(удовольствие и боль) и косвенные(). Дружба,
  симпатия -- говорил об этом. Религия основана на инстинктах.
\end{itemize}
\end{itemize}
\end{document}
