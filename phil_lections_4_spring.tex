\documentclass[a4paper,12pt]{report} %размер бумаги устанавливаем А4, шрифт 12пунктов
\usepackage[T2A]{fontenc}
\usepackage[utf8]{inputenc} %включаем свою кодировку: koi8-r или utf8 в UNIX, cp1251 в Windows
\usepackage[english,russian]{babel} %используем русский и английский
                                %языки с переносами
\usepackage[pdftex,unicode]{hyperref}
\usepackage{amssymb,amsfonts,amsmath,mathtext,cite,enumerate,float} %подключаем нужные пакеты расширений

\usepackage{geometry} % Меняем поля страницы
\geometry{left=1cm} % левое поле
\geometry{right=1cm} % правое поле
\geometry{top=1cm} % верхнее поле
\geometry{bottom=2cm} % нижнее поле

\renewcommand{\theenumi}{\arabic{enumi}} % Меняем везде перечисления на цифра.цифра
\renewcommand{\labelenumi}{\arabic{enumi}} % Меняем везде перечисления на цифра.цифра
\renewcommand{\theenumii}{.\arabic{enumii}} % Меняем везде перечисления на цифра.цифра
\renewcommand{\labelenumii}{\arabic{enumi}.\arabic{enumii}.} % Меняем везде перечисления на цифра.цифра
\renewcommand{\theenumiii}{.\arabic{enumiii}} % Меняем везде перечисления на цифра.цифра
\renewcommand{\labelenumiii}{\arabic{enumi}.\arabic{enumii}.\arabic{enumiii}.}
% Меняем везде перечисления на цифра.цифра


\begin{document}
\begin{itemize}
\itemГреция
  \begin{enumerate}
  \itemЭпикур
    \\атараксия(что-то вроде Дао)
    \\детерминизм
    \\сенсуалист
    \\борьба со страхами(страхом смерти, например)
  \itemСтеицизм(основатель --- Зенон 4~--~3 вв д.н.э.)
    \\Философия --- яйцо
    \\Логика --- скорлупа
    \\Физика --- белок
    \\Этика --- желток
    \\Сенсуалист
    \\Мир пронезан пневмой(божественным духом), мир и Б-г --- одно и то
    же(пантеизм).
    \\Фаталисты
    \\Мир имеет цель(неизвестую), понять ееможно с помощью философии
    \\Пневма по сути --- Логос
    \\Теодицея --- оправдание зла. Почему оно существует?
    \\Если бы не было зла, мы бы не замечали добра.
    \\То, что кажется нам злом, может злом не являться.
  \end{enumerate}
\itemРим
  \begin{enumerate}
  \itemСенека(\textbf{Стеицизм})(1-й в н.э.)
    \\Проповедовал бедность и осуждал стремление к богатству
    \\Его мало интересуют знания о мире
    \\Допускал существование богов(боги телемны, все одушевлено, разумно,
    божественно)
    \\Воспринял учение о том, что все состоит из элементов
    \\Самоубийство занимает не малое место в его философии
    \\Идея равенства: рабство --- позорно
    \\Величайшее дело жизни: противостоять ударам судьбы
    \\Греховность человека заложена изначально
    \\"Душа --- это бог, нашедший приют в теле человека''
  \itemЭпиктет(1~-~2 в н.э.)
    \\раб стал философом
    \\измени отношение к вещам, если не можешь изменить эти вещи
    \\с отношением к богам не определился
    \\"Выдерживай и воздерживайся
  \itemМарк Аврелий(160~-~180 гг)
    \\Скоротечность существования
    \\Воспринимать все стоически и при этом быть деятельным и энергичным
    человеком
    \\Что-то делать во благо других
    \\Человек тройствинен: тело(бренное), душа(жизненная сила),
    разум(руководящее начало)
  \itemХристианство (Патристики)
    \\Монотеизм
    \\Бог творец, Бог активен
    \\Нет однозначного отношения к материи(не принижается плоть)
    \\Заповеди
    \\313 г.н.э., император Константин прекратил гонения христиан
    \\325 г. Никейский Собор, формулирование символа веры
    \\Роды Христа(арианство(ересь) -- Христос хороший человек, проповедник, не
    Бог, монофизиты(еретики) -- считавшие, что у Христа только
    божественная сущность)
    \\Троица, единство Бога
    \\Божественная природа Богоматери
    \\Была ли воля у Христа? (монофилиты -- только воля Бога, у христа не
    было выбора)
  \itemГностицизм --- соперники христианства(опор на знания)
    \\Три категории людей:
    \begin{itemize}
    \itemПневматические(дух, духовный, обладающие сакральным знанием)
    \itemПсихические(душа, душевный, хороший)
    \itemГилические(материальные)
    \end{itemize}
  \itemМетроизм 
  \itemАпологеты --- защитники христианской церкви
    \begin{enumerate}
    \itemКлимент Александрийский(2~-~3 в н.э.). Терпимость, отвергать
      ничего не стоит(возьмем что-то из античной традиции), но вера прежде
      всего.
    \itemТертуллиан(2~-~3 в н.э.) --- идея патриотизма ему чужда. Критиковал Римскую
      культуру. ``Веруй ибо абсурдно''.
    \itemОриген(2~-~3 в н.э.) --- еретик. Родился в семье проникнутых
      христиан. Изучал священные тексты. У Амония(учился) застал
      Платина. Экзегетика -- искусство толкования священных текстов.
      \\Три уровня понимания библии:
      \begin{itemize}
      \itemБуквальный уровень(Тертуллиан) -- по Оригену соответствует телу
      \itemНравственный смысл -- соответствует душе
      \itemДуховный -- способны постичь только избранные
      \end{itemize}
      ``В конечном итоге спасутся все --- даже дьявол''.
    \end{enumerate}
  \itemАврелий-Августин(блаженный)(4~-~5 в н.э.) --- епископ. Большое влияние на
    него произвела книга Цицерона(исповедовал манихейство -- оправдание
    греха).
    \\Его произведения:
    \begin{enumerate}
    \itemИсповедь(новый жанр)
    \itemПротив манихейства
    \itemПротив донатизма(донатисты за нравственность священников,
      Августин считал, что они все ответят сами)
    \itemО граде божьем(аллегория град божий(Иерусалим), град
      земной(Вавилон)) Я сам для себя стал проблемой -- я наблюдатель и
      наблюдаемый(человек -- чудо).
    \end{enumerate}
    \\Античное наследие -- ок. Любил Платона.
    \\Вопрос веры. Приоритет веры над знанием. Разум тоже важен.
    \\"Не дай мне Бог сойти с ума''. Сомневаетесь -- значит,
    существуете(предшественник Декарта).
    \\Благодать. Выход из безвыходной ситуации.
    \\Грех. Теодицея(оправдание бога) -- оправдание зла. То, что кажется
    злом, на самом деле им не является. Причина зла -- ваши собственные
    поступки. Зло -- это расплата за первородный грех.
    \\Признает значение науки. Делит науку на полезные(математика, языкознание) и
    бесполезные(астрология, магия, театральная наука).
    \\Живи настоящим. Хронос(время течет), кайрос(шанс, единомоментный).
  \end{enumerate}
\itemСредневековье
  \\476 г.н.э. -- падение Римской Империи. Синтез варварской и античной
  культуры. 6 в. -- формирование государства(королевства), начались
  вассальные отношения. 
  \begin{itemize}
  \itemСимволизм, приверженность
    ритуалу. 
  \itemИерархизм. 
  \itemУниверсализм, стремление постичь мир в
    целостности. 
  \itemТрадиционализм, каноничность.
  \itemСинтез христианства и язычества
  \itemСхоластика(школа, обучение). Представители схоластики
    обосновывали религиозные истины, опираясь на священное писание и
    труды Аристотеля. Первый университет(12 в) -- Болонский
    Университет. Оксфорд (13 в).
  \end{itemize}
  \begin{enumerate}
  \itemДионисий Ареопагит -- разработал иерархию ангельских чинов.
  \itemПьер Абеляр (1079~-~1142). ``История моих бедствий''.
    \begin{enumerate}
    \itemРационализм. Понимай, чтобы верить.
    \item Что мешаетпостичь священное писание
      \begin{enumerate}
      \itemОграниченность знаний историческая
      \itemЗнания должны приращиваться
      \end{enumerate}
    \itemЭкзегетика. Как приблизиться к истине? 
      \\Нужно определиться с терминами. 
      \\Четко установить аутентичность текста(авторство).
      \\Сверить текст с другими произведениями автора
    \itemИнтенция -- осознаный умысел.
    \itemНакопление богатства -- плохо
    \itemИлаиза: ``Без рацио библия все равно, что зеркало для слепых.''
    \end{enumerate}
  \itemФома Акминский (13 в) -- доминиканский орден(состоял в нем).
    \begin{enumerate}
    \itemТеология -- истинное откровение, философия -- истинныйразум. Философия
      должна подчиняться теологии. Божественные знания через откровения. 
      \\4 ступени истины:
      \begin{enumerate}
      \itemопыт(империо) 
      \itemискусство(техне)
      \itemзнание(эпистема)
      \itemмудрость(софия)(благодать).
      \end{enumerate}
      \\Истинные откровения: доступные понимаю(что можно постичь),
      недоступное пониманию(истина откровения).
      \\Проблема сущности и существования
      \\Человек -- единство души и тела. Он стоит между ангелом и животным.
      \\Воля следует интеллекту.
      \\Универсалии: номиналисты и реалисты. У реалистов считается, что
      общие понятия существуют, а у номинолистов, что вот человек
      существует, а общего нет.
      \\Фома считает, что универсалии существуют(до вещей -- в божественном
      разуме, в самих вещах -- в их сущности и в мыслях человека).
      \\Интеллект делится на активный(понятия) и пассивный(чувства,
      образы).
      \\Чувства: внутренние(память) и внешние(зрение, слух, обоняние,
      осязание).
      \\Созерцание, суждение, умозаключение --- 3 способа(этапа) познания.
      \\3 вида знания: ум, духовные способности, интеллект
    \end{enumerate}
  \end{enumerate}
\itemЭпоха возрождения
  \begin{itemize}
  \itemПантеизм(человек не был атеистом)
  \itemАнтропоцентризм(человек ощущает себя более самодостаточным) 
  \itemгармония(изменилось отношение к
    человеческому телу, иделом становится разносторонне развитый человек,
    должен был быть красивым внешне и внутренне)
  \itemГуманизм. Многие знали греческий, латынь(было престижно знать
    языки, заниматься переводами). Искусство -- круто. \textbf{Возвращение к
      античности. Как к трудам авторов, так и к античному искусству.} В
    каком-то смысле возрождают понятия эпикуристов(духовное
    наслаждение). Сочетание идеалов жизни созерцательной и активной --
    не все в руках Бога, человек сам творит свою судьбу. Модно обсуждать
    достоинства человека(добродетели). Нравственный(мужество, щедрость,
    великодушие), умственные(мудрость, благоразумие, способность к науке
    и искусству). Фатум(человек полностью зависит от Бога),
    фортуна(шанс, способность влиять на вещи). Петрарк. Люди начинают
    обустраивать свое жилище, включаются в политическую жизнь.
  \itemВозвращаются мифы. Например, возраждается орфизм,
    герметизм. Занятия магией популярны(даже средилюдей, находящихся в
    лоне церкви).
  \end{itemize}
  \begin{enumerate}
  \itemНиколай Кузанский (1401~-~1464).\\
    Родился в Кузе. Сбежал из
    дома. Закончил университет. Дослужился до кардинала. Был папским
    регатом(советник по особым делам).
    \begin{enumerate}
    \itemПапские притязания на верховную власть безосновательны.
    \itemПытался примирить церковь. Толерантность.
    \itemИдея Бога. Пантеизм -- Бог это все, субстанция,
      котораяпронизывает мир. Присутствует иерархия: Бог -> Ангелы ->
      Человек -> Форма -> Материя(неоплатонизм). Мир сотворен Богом(с
      помощью арифметики, астрономии и всех
      искусств...). Универсалии(только одна -- Бог) в нашем
      сознании(концептуализм). Человек -- микрокосм, мир -- макрокосм. 
    \itemУделял большое внимание математике. Пытался описать идею Бога с
      помощью математики.
    \end{enumerate}
  \itemМарсилио Фичино (1433~-~14).\\
    \textbf{Переводы, философия,
    магия.} Потылся примирить античность с христианством. Смысл занятий
    философией в том, чтобы приготовить душу к восприятию божественного
    откровения. Все проистекает из Логоса(неоплатонизм). Магические
    законы основаны на природных законах -- естесственная магия. Думал
    об объединении церкви.
  \end{enumerate}

\end{itemize}
\end{document}

