\documentclass[a4paper,12pt]{report} %размер бумаги устанавливаем А4, шрифт 12пунктов
\usepackage[T2A]{fontenc}
\usepackage[utf8]{inputenc} %включаем свою кодировку: koi8-r или utf8 в UNIX, cp1251 в Windows
\usepackage[english,russian]{babel} %используем русский и английский
%языки с переносами
\usepackage[pdftex,unicode]{hyperref}
\usepackage{amssymb,amsfonts,amsmath,mathtext,cite,enumerate,float} %подключаем нужные пакеты расширений

\usepackage{geometry} % Меняем поля страницы
\geometry{left=1cm} % левое поле
\geometry{right=1cm} % правое поле
\geometry{top=1cm} % верхнее поле
\geometry{bottom=2cm} % нижнее поле

%\renewcommand{\theenumi}{\arabic{enumi}} % Меняем везде перечисления на цифра.цифра
%\renewcommand{\labelenumi}{\arabic{enumi}} % Меняем везде перечисления на цифра.цифра
%\renewcommand{\theenumii}{.\arabic{enumii}} % Меняем везде перечисления на цифра.цифра
%\renewcommand{\labelenumii}{\arabic{enumi}.\arabic{enumii}.} % Меняем везде перечисления на цифра.цифра
%\renewcommand{\theenumiii}{.\arabic{enumiii}} % Меняем везде перечисления на цифра.цифра
%\renewcommand{\labelenumiii}{\arabic{enumi}.\arabic{enumii}.\arabic{enumiii}.}
% Меняем везде перечисления на цифра.цифра


\begin{document}
\begin{itemize}
\item \textbf{Эпоха возрождения}
  \\ 
  \textit{Основные черты: } Античность(искусство, внутренний мир человека, человеческое тело,
  человеческий дом, пантеизм(Бог тождественен природе), языки(греческий,
  латынь)), астрология,магия, алхимия.
\item Никколо Макиавели (1469 - 1527). Флоренция. \\
  Взгляды:
  \begin{itemize}
  \item Политика: смотреть на жизнь реально. Люди ни хорошие, ни плохие,
    но больше плохие. Люди ценят силу. Вирту(добродетель) имел в разное
    время разное значение. У Макиавели вирту - это сила. Цель
    оправдывает средства для государя. Правитель должен убрать все, что
    было до него.
  \item Бог это фортуна, фортуна это женщина, женщины любят силу, нужно
    брать ее в свои руки и управлять ей. Религия нужна. Она как цемент
    для государства.
  \end{itemize}
\item Томас Мор (1478 - 1535). \\
  Утопия - проект политического преобразования, за основу взяты идеи
  Платона. Это всего лишь модель. Не претендует на реализацию в
  жизни. Частная собственность вредит. Большая роль отводилась религии,
  но в государстве провозглашена свобода вероисповедания. Рабочий день
  составлял 6 часов. От каждого по способностям, каждому по
  потребностям. Имело место рабство. Ручной труд.
\item Томмазо Компанелла (1568 - 1639). \\
  Отсутствие частной собственности. Делал ставку на технический
  прогресс, каждый житель должен обучаться ремеслу. Нет негодяев и
  тунеядцев, все работают, каждый обучается военному делу. Люди
  обмениваются домами каждые полгода. Человек должен трудиться 4 часа в
  день, а в последствии еще меньше. Управляется государство верховным
  жрецом (метафизиком). У него три жреца в подчинении: могущество(война и мир), мудрость (наука и
  просвещение), любовь(устройство браков). Селекционирование людей. Если
  люди полюбили друг друга, то они должны были спросить разрешения у
  жреца. Если жрец против, то они могли дарить друг другу венки и писать
  стихи. Вознаграждение определяется начальником. Религия существует. Но
  странная. Большую роль играет астрология. Люди с ограниченными
  возможностями тоже могут заниматься делами.\\
  \textbf{Гносиология (познание), сенсуалисты(эмпирики)(в основе познания лежат чувства) и
    рационалисты(первична мысль, разум человека). Метод индукции(от
    частного к общему) - эмпирики, рационалисты - дедукция. Метод
    монументальной пропаганды (плакаты).
  }
  Компанелла эмпирик, но считает, что разум важен. Потому что есть
  галлюцинации, сны и т.д. Сущность нам недоступна, мы можем познавать
  только явления.
\item Мартин Лютер \\
  Вывесил 95 тезисов на дверях университета. Они очень быстро
  распространились по всей Германии.\\
  \begin{enumerate}
  \item Церковь не является посредником между Богом и землей.
  \item Священник: крещение, отпевание
  \item Священник объясняет людям библию
  \item Принцип спасения верой - человек спасается личной верой, а не
    выполнением церковных обрядов. Верующий человек не может совершать
    плохие дела.
  \item Отменить все виды оплаты.
  \item Закрыть монастыри - аскеза не нужна, нужна сдержанность.
  \item Целебат(безбрачие) отменяется
  \item Никаких индульгенций
  \end{enumerate}
  Протестанты Франции, северных стран, Германии, Испании, Италии,
  Польши, Швейцарии быстро узнали эту доктрину. А в Швейцарии был
  кальвенизм. Жан Кальвин. Принцип божественного предопределения: Бог уже
  знает, кто попадет в ад, а кто в рай. Чтобы выяснить куда попадешь
  нужно неустанно работать. Если ты богат, то попадешь в рай. Но при
  этом все жили в аскезе. Никакой роскоши, все дома открыты,
  обязательное посещение проповедей Кальвина. Так получилось, что
  протестантские страны стали богаче остальных.\\
\item Коперник, Николай (1473-1543)\\
  Астрономия была его хобби, он был юристом(каноником - церковным
  юристом), монахом. Так же увлекался медициной и лечил бедняков
  бесплатно. Отразил нападение Тевтонского Ордена, провел денежную
  реформу, боролся с эпидемиями, участвовал в дипломатических миссиях. \textbf{НИКТО ЕГО НЕ ЖЕГ НА КОСТРЕ.}\\
  Написал работу ``Об обращении небесных сфер''. Церковь на эту работу
  никак не отреагировала. Работа была признана еретической только
  1616г.\\
  Ближе к неоплатонизму, путь солнца, герметические учения. Бог --
  геометр, создал мир в соответствии с математикой. ``Мир
  сферичен''. \textit{Гелиоцентрическая система. Земля вращается вокруг Солнца и
    вокруг своей оси}.
\item Бруно, Джордано (1548-1600)\\
  При рождении получил имя Филипп. Когда стал монахом, ему дали имя в
  честь реки Иордан. По слухам убил монаха в монастыре и ему пришлось
  бежать. Путешествовал по городам Италии, побывал в Швейцарии. А в
  Швейцарии были кальвинисты уже, он с ними подружился, а потом
  рассорился и отправился во Францию. Познакомился с королем Генрихом
  3. Отправился в Лондон, где познакомился с Оксфордскими
  преподавателями, назвал их педантами и они обиделись. Назвали его
  плагиатором. Вернулся в Париж и познакомился с последователями
  Аристотеля(схоластами). Поссорился с ними и бежал в
  Германию. Познакомился с лютеранами, вступил в их общину и был
  изгнан. Вернулся в Италию, где его попросили заняться преподаванием
  тренеровки памяти. Вспылил и оскорбил ученика, а тот написал на него
  донос. Утверждал, что Иисус просто человек, исповедовал идею
  реинкарнации, говорил, что небесные тела одухотворены и поклонялся
  Солнцу, был пантеистом. \underline{Миров много}. Гелиоцентризм объяснялся
  культом Солнца. Звезды -- это солнца.

  Занимался тренировкой памяти и разработал доктрину для развития
  памяти. В основе памяти лежат архетипы, с помощью которых можно
  обрести космическую силу.
\item Тихо Браге (1546-1601)\\
  Происходил из знатной семьи. Поступил в Копенгагенский университет в
  12 лет. Учился на юриста. Его поразило затмение Солнца и побудило
  заниматься астрономией. В 16 лет уехал учиться в Германию на 6
  лет. Иногда заезжал в Данию. Видел соединение Юпитера с Сатурном, что
  тоже его поразило. Обладая острым зрением, измерения проводил с
  помощью одного лишь циркуля. Поссорился с другом, был вызван на дуэль,
  где ему шпагой отсекли часть носа -- был вынужден носить серебряные
  протезы. Из-за этого у него развились комплексы и он никогда не
  женился. После смерти отца наладил стекольное производство(был
  алхимиком и астрологом). И занимался астрономией. Наблюдал сверхновую
  и написал целое произведение ``О новой звезде''. Вспыхнула она ровно
  через месяц после Варфоломеевской ночи и начались толки о конце
  света. Фридрих 2 заметил Тихо Браге и дал ему в пожизненное владение
  остров. Там Браге построил обсерваторию и назвал ее небесный
  замок. Достиг фантастической точности в своих наблюдениях. \underline{Он выяснил,
    что небесных сфер нет}. Подсчитал все небесные тела и составил
  глобус. Количество звезд оценил где-то тысячью. Думал, что Земля
  неподвижна. В конце жизни встретился с Кеплером. Они поссорились и
  Браге вскоре умер. Он хотел, чтобы Кеплер стал его приемником.
\item Кеплер, И. (1571-1630)\\
  Сын служителя лютеранской церкви. Родился болезненным, переболел оспой
  в детстве. Из-за болезней в детстве у него было слабое зрение. Его
  мама подсыпала в напитки посетителям трактира разные травы и наблюдала
  за их поведением. Поступил в университет, а хотел стать
  священнослужителем. Но в 22 года он передумал и начал преподавать
  математику и делал прогнозы различных событий(волнения крестьян,
  осадки). Переписывался с Галилеем. Кеплер женился на богатой вдове,
  император назначил его придворным математиком. Получал немалые деньги
  от императора.\\
  Занимался проблемой света. \underline{Изображение попадает на сетчатку в
    перевернутом виде.}\\ \underline{О стереометрии, вычислял объем бочки методом,
    близким к дифференциальному.}\\ \underline{Законы Кеплера.} Делал все это, как
  маг. Мир описывается математическими законами. Ощущения дают
  хаотическую массу информации, которую обрабатывает разум. К учению
  Коперника относится крайне восторженно.
\item Галилей, Галилео (1564-1642)\\
  С детства проявил склонность к конструированию. Интересовался
  медициной, богословием. Любил пребывать в изуитском монастыре, но
  монахом не стал. Пригласили преподавать медицину ординарным(деньги
  платит университет)
  преподавателем, но он стал преподавать математику
  неординарным(деньги платили студенты)
  преподавателем. Его интересовала теология, изучал круги ада
  Данте. Уезжает преподавать в Голландию. Там пишет большую часть своих
  работ. Там он познакомился с телескопом. Нашел патрона, который ему
  покровительствовал. Увлекся идеями Коперника. Его защищал Папа, но
  Гилилей написал произведение, высмеивающее церковь и Папа
  обиделся. Была у него дочь - монахиня, он ездил к ней в
  монастырь. Когда она умерла, он ``расклеился''. Его сожгли.\\
  Устройство космоса по Аристотелю: подлунный и надлунный мир, небесные
  сферы, Бог -- перводвигатель. Галилей \textit{сделал супер-телескоп x30} и сказал,
  что \underline{нет никакого надлунного мира}. Исповедует идеи Коперника,
  ставит Солнце в центр мира.\\ \underline{Разработал идею
    математического(мысленного) эксперимента, что бы было в идеальных
    условиях.}\\ \underline{Теория двойственной истины. Наука и вера несовместимы.}\\ \underline{Бог
    дал нам разум, чувства.}\\ \underline{А знания получаем посредством чувственного
    опыта.}\\ \underline{А разум дан, чтобы знания удостоверить.}\\ \underline{Галактика -- скопление
    звезд.}\\ \underline{Священное писание -- не трактат по астрономии, там не нужно
    искать объяснения творения мира.} Священное писание не учит жить на
  земле, а учит как душе попасть в рай. Экзегетика(толкование
  библии). \textit{Писание правильно истолковать может только ученый.}
  Не следует искать в писании ответы на вопросы, на которые человек
  может ответить своим разумом. \textit{Наука нейтральна к миру ценностей, вера
    некомпетентна в вопросах факта. Какова сила истины: вы пытаетесь ее
    опровергнуть, но сами ваши нападки возвышают ее и придают ей большую ценность}.
\end{itemize}
\begin{enumerate}
\itemНаучная революция (от создания Коперником ``Обращение небесных сфер''
  до ``Начала'' Ньютона). Создание Гелиоцентрической системы.
\itemУниверситеты(схоластика) уступили место академиям(научный метод:
  наблюдение, эксперимент).
\itemОживление контактов между учеными. Распространение знаний с
  помощью книгопечатания.
\itemУченые не были атеистами, идея Бога видоизменяется. Наука и
  христианство соседствуют с магией, астрологией, оккультными
  практиками.
\itemУченый -- новый облик, человек, который в основу кладет
  эксперимент. Все инструменты и эксперименты делали сами.
\itemСближаются ученые и ремесленники. Ученый -- человек деятельный.
\end{enumerate}
Что-то новенькое
\begin{itemize}
\itemБэкон Ф. (1561-1626)\\
  Родился в богатой, знатной семье, в Англии. Отец был хранителем государственной
  печати, мать переводила с латыни. Устроился на гос. службу. Начал
  выступать против политики Елизаветы. Но позже Яков его приблизил к
  себе и покровительствовал ему(Бэкону).\\
  Считается родоначальником современной философии(\textbf{эмпирики}).\\
  Аристотеля не уважал, вообще не очень относился к древним
  философам. Бэкон говорит, что философия древних -- это детство, нужно
  все переписывать. Так появились ``Новая Атлантида'' и ``Новый
  органон''. Теперь во главу угла ставится метод. Бэкон: ``Знание --
  сила''. В царство науки нужно входить чистым, как дитя.\\
  Препятствия на пути к познанию:
  \begin{enumerate}
  \itemИдол рода(органы чувств)
  \itemИдол пещеры(свои заморочки, привычки)
  \itemИдол рынка(слухи)
  \itemИдол театра(капание на мозги, типа рекламы)
  \end{enumerate}
  Предлагает отказываться от авторитетов. Истина -- дочь времени, а не
  авторитета. Причины заблуждений: софистика, эмпирика(опыт
  недостаточен), суеверия.\\
  Атеизм -- тонкий лед, по которому один человек пройдет, а нация
  провалится.\\
  Индукция -- от частного к общему. Индукция это метод Бэкона. Таблица
  присутствия, отсутствия и степеней.\\
  Муравьи -- эмпирики, собирающие знания, пауки -- рационалисты,
  развивающие идеи только из своей головы, пчелы -- ученые.\\
  Плодоносные опыты -- дают возможность использовать результат в будущем, светоносные опыты -- помогают открыть скрытые
  причины явления.\\
  Классификация наук:\\
  В основе -- способности человека:\\
  Память, воображение, рассудок(философия: физика и метафизика).\\
  Науки: прикладные(физика -- механика, метафизика(магия)) и
  теоретические.\\
  Стал жертвой опыта: проводил опыт с замороженной курицей, простудился
  и умер.
\itemЛокк Д. (1632-1704)\\
  Обучался в хорошей школе, несмотря на то, что был небогат. Отказался
  от Аристотеля. Был хорошим врачом и служил у знатного лорда Эшли. Был
  принят в Лондонское Королевское Общество. В конце жизни поселился в
  замке знатной дамы.\\
  Гносиология. Сенсуалист -- в основе познания лежат чувства. Сознание
  ребенка -- чистая доска. Идея(из внешнего мира, рефлексивные(путем
  размышлений)). Рефлексии вторичны -- возникают на основе чувственного
  опыта. Идеи сложные: путем суммирования, сопоставления, выделение
  общих черт через абстракцию. Каждому предмету присущи первичные
  качества, которые реально существуют в телах(вес, форма), и
  другие(цвет, вкус).\\
  Чувственное познание -- самое достоверное. Демонстративное
  познание(разум) -- умозаключения. Интуитивное познание -- озарение.\\
  В государстве должно существовать разделение властей(исполнительная,
  законодательная). Существуют естественные права. Свобода совести --
  это свобода вероисповедания.\\
  Локк говорит, как надо воспитывать детей. Надо воспитывать ребенка
  целенаправленно, разностороннее обучение ему давать.
\itemГоббс Т. (1588-1679)\\
  Окончил Оксфорд(поступил в 15 лет). Отлично знал латынь. Устроился
  воспитателем к богатому джентльмену, путешествовал вместе с ним.\\
  Гносиология. Гоббс был человеком с крупным телом. В трактате ``О
  теле'' он пишет, что тела бывают естественные(материальные) и
  искусственные(государство). С помощью законов физики можно описывать и
  политику. Природа -- совокупность протяженных тел, различающихся
  величиной, фигурой. Неотъемлемые свойства -- протяженность и
  фигура. Остальные качества -- фантомные. Пространство -- это
  воображаемый образ существующей вне нас вещи, а время -- это образ ее
  движения. Теория языка, как знаковой системы. Истина -- понятие
  абстрактное, о ней говорить бессмысленно, научное знание
  относительно. Нужно изучать только реальные вещи. Гипотеза -> дедукция
  должна сосуществовать с индукцией. Подчеркивал значение интуиции.\\
  ``Левиофан'' -- государство. \underline{Теория общественного
    договора}.\\
  Деятельность человека невозможна без осознания им своей
  свободы. Естественные права. Идеальная форма правления -- монархия.\\
  Соперничество, недоверие и жажда славы.\\
  Деизм(Теизм?) -- Бог создал мир и удалился. Религия должна быть
  подчинена государству. Убрать мистику, астрологию, магию из
  религии. Религия должна быть удобной для государства. Главное --
  практическая выгода, он был прагматиком.
\itemБеркли, Джордж (1685-1753)\\
  Семья среднего достатка. Стал англиканским священником после
  колледжа. Стал профессором и преподавал языки. Двинул в
  Америку. Познакомился со Свифтом. Вернулся в Лондон, стал деканом. ``О
  пользе дегтярной настойки''.\\
  \underline{Существовать значит быть воспринимаемым}. Беркли не нравился
  материализм. Слова служат для обозначения не реальных предметов, а
  наших идей. Идеи возникают на основе ощущений или рождаются в нашем
  разуме, все это в результате восприятия мира. Первичных качеств нет --
  все субъективно. Не отрицал существование мира, но о материи говорить
  бессмысленно, потому что сам термин придумал человек. Реляционная
  концепция пространства-времени. Отрицал абстрактные идеи. Он был
  \underline{номиналистом}. Все предметы существуют изначально в
  сознании Бога, а он развил их в сознании человека. 
\itemЮнг(ктоблять?), Дэвид (1711-1776)\\
  ``Наука о человеческой природе''. Книгу не заметили и у него начался
  духовный кризис. Он очень хотел стать известным и стать профессором
  в Оксфорде, куда его не пускали из-за плохого отношения к
  религии. Пытался подружиться с Руссо. Юнг помогал материально Руссо
  и Руссо подумал, что Юнг хочет его уничтожить. Написал трактат по
  истории и его заметили, начали считать крутым историком. Прежде чем
  начинать процесс познания, нужно понять природу человека, как мы
  познаем мир. Агностицизм -- мир познать невозможно. Мы не можем
  ничего говорить о мире с полной уверенностью, потому что все это --
  наше впечатление. Впечатления делятся на простые и сложные. Цвет,
  тепло, форма -- простые, цветок -- сложное. Восприятие формирует
  идеи, память дает возможность создавать идеи. Человек -- чистый
  лиц. Свобода. Номинализм. Страсти и аффекты. Страсти
  делятся на прямые(удовольствие и боль) и косвенные(). Дружба,
  симпатия -- говорил об этом. Религия основана на инстинктах.
\itemДекарт, Рене\\
\itemЛейбниц  (1646-1716)\\
  Родился в семье интеллигентов. Участвовал в дипломатических
  миссиях. Убедил Людовика воевать не с Германией, а с Османской
  империей. Работал у богатого человека и путешествовал с ним. Лейбниц
  стал заведующим библиотекой. Лейбниц подсказал идею коллегий Петру
  I. Много уделял внимания религии, призывал к терпимости. Отказывался
  писать на немецком языке, потому что он еще не сформирован. Писал на
  латыни и французском. Последние годы жизни прожил в нищете и
  одиночестве.\\
  Все состоит из субстанций(монад). Монады имеют свои цели, свои
  значения, они активны, не имеют пространственных свойств, зато они
  имеют духовную силу. Четкой границы между духовной и вещественной
  частями природы, у всего есть душа. Три типа монад: простые(растения,
  способность восприятия), души(животный, свойство памяти), духи(человек,
  разум, самосознание), бог. Был деистом. Бог заложил принцип работы в
  монады. Мы живем в лучшем из миров, потому что бог оставил мир в
  гармонии. Бог заложил не сами идеи, а способность мыслить. Отрицает
  идею чистой доски. Нет ничего в разуме, что прежде не было в чувствах,
  кроме самого разума. Концепция двух истин: истинный разум(всеобщие
  истины, вечные), истинный факт(сиюминутные знания). Предложил идею
  времени-пространства. Без материи нет пространства. Пространство не
  представляет абсолютной реальности, как и время. Большое внимание
  отводил логике. Пытался объединить религии.
\itemНьютон, Исаак (1643-1727)\\
\item *Blanc*
\itemВольтер (1694-1778)\\
  Создавал исторические труды, (а не только оды, восхваляющие французского короля). Ход
  истории не подвержен божественному промыслу. Раньше было принято
  писать историю правителя, а он писал историю цивилизации, чтобы
  просматривать динамику ее развития и гибели. ``Раздавить гадину'' -
  это было адресовано церкви, как социальному институту. ``Церковь
  возникла тогда, когда встретились мошенник и глупец''. ``Атеизм и
  фанатизм - два полюса ужаса''. Не был атеистом, был деистом. Тем не менее, на
  смертном одре он попросил призвать к себе священника. Проблема
  познания: сенсуалист. Сознание - атрибут материи и зависит от строения
  тела. Проблема свободы воли: нерешительно, но последовательно
  переходил от индетерменизма к детерминизму (все в жизни имеет свои
  причины). К политикам относился скептически. ``Политики знают о делах
  своих министров не больше, чем рогоносец о делах своей жены''. ``Свобода -
  независимость от всего, кроме закона''. К Паскалю и Лейбницу относился
  скептически. Человек не тростник(Паскаль), мы не живем в  лучшем из
  миров(Лейбниц). Государство не должно пытаться дать образование всем.
\itemСмит, Адам (1723-1790)\\
  Родился в небогатой семье. Его воспитывала молодая мать. Она очень
  много занималась своим ребенком. Преподавал разные дисциплины:
  риторика, языкознание и т.д. Его работы по экономике стали
  читать. Деньги с этих статей потратил на благотворительность. Не
  женился.\\
  Стремление к благосостоянию это не плохо (разумный
  эгоизм). Оправдывает стремление к благу. ``Невидимая рука рынка
  отрегулирует все''. Во время его жизни произошел промышленный
  переворот (переход от мануфактуры к фабрикам). У кого есть машина, тот
  будет производить больше. Свободная торговля (отсутствие пошлин и
  т.п.) важна. Затронул проблему налогообложения. Налоги должны быть
  необременительными должны взиматься вовремя. Теория абсолютного
  преимущества. Страна должна специализироваться на производстве того,
  что она может производить наиболее эффективно (нет смысла развивать
  другие отрасли). Сократить количество рентодержателей, юристов,
  чиновников, военных.
\itemРикардо, Дэвид (1772-1823)\\
  Был практиком. Родился в семье биржевого маклера. Был четвертым из 17
  детей. С детства работал с финансами. К 16 годам не имел
  систематического образования, но мог самостоятельно справляться с
  делами отца. В 21 год решил жениться. Но он был из семьи
  ортодоксальных евреев, а девушка была христианкой. Родители не дали
  благославления. Денег ему не дали. Он женился. Но благодаря своим
  навыкам он поднялся и смог в конечном итоге содержать 8 детей. Жил в
  аристократическом квартале Лондона, имел личную загородную
  резиденцию. Изучил труды Смита и решил написать свой. Его книги тоже
  скоро имели успех.\\
  Делит всех на три класса: рабочие (труд, зарплата), предприниматели
  (прибыль), землевладельцы (рента). Плодородность земель уменьшается, а
  рентодержателей все больше и больше. А они ничего не создают. Теория
  сравнительных преимуществ. Страна должна производить то, что ей
  выгоднее производить, чем другой стране. Если рабочим повышать
  зарплату, то они будут заводить больше детей, что увеличит количество
  новых рабочих, что приведет к сокращению зарплаты рабочим.
\itemМальтус, Томас\\
  Небогатая семья. Сам был бедный. Протестант. Священник.\\
  Мальтузианские проблемы. Количество людей растет в геометрической
  прогрессии, а количество ресурсов в арифметической, поэтому всем хана
  скоро. Людям нужно вступать в брак позже и ограничивать себя в
  создании детей. Нет поддержки малоимущим. Нужно помогать
  предпринимателям, но рынок сам все должен отрегулировать (в том числе
  количество рабочей силы).
\itemКант, Имануил (1724-1804)\\
  Родился слабым, говорили, что не выживет. В бедной семье. Родители были
  пиитистами (протестанты). Отец рано умер. Мать брала с собой к священнику Шульцу
  сына. Тот, заметив способности мальчика, отправил его в школу. Получил
  там фундаментальное образование. Поступил в университет. Прекратил
  обучение из-за отсутствия денег. Пошел работать домашним
  учителем. Позже защитился, преподавал в университете и в последствии
  стал ректором. Всеобщая естественная история и теория
  неба. Космогоническая гипотеза там была: Солнечная система возникла из
  газовой туманности. Был неординарным преподавателем. Был беден и не
  имел своего дома. Потом его все-таки перевели в штат, он сумел
  приобрести себе свой дом. У него даже был свой штат слуг. Дом
  соседствовал с тюрьмой и постоянно слышал пение заключенных. Это
  раздражало Канта, потому что он не мог работать. У Канта был очень
  четкий распорядок дня, он был очень пунктуальным. Кант был гурман, до
  мелочей следовал моде. На обед приглашал совершенно разных людей,
  чтобы проводить его за беседой. Учил на дому. Выступал с инициативой
  защищать студентов от армии. Писал даже о Сибири.\\
  Немецкая классическая философия. В творчестве выделяются два периода:
  докритический (естественные науки) и после. ``Критика чистого
  разума''. Критика - это анализ. Чистое - без опыта.\\
  Трансцендентное - недоступное нашему пониманию.
  Трансцендентальное - недоступное, но мы можем приходить к выводам
  каким-то.
  Трансцендентальная эстетика ставит задачу привести в порядок
  чувства. Эстетика - то, что базируется на наших чувствах.\\
  критическая философия.\\
  /\
  теория - что я должен знать | практика - что я должен делать, на что я
  могу надеяться -> императивы (категорический (принципы, не нарушаемые
  ни при каких обстоятельствах) и гипотетический (правила, которые можно
  нарушить в каких-то критических ситуациях)).\\
  Вещь в себе. Мир непознаваем. Мир является нам сквозь призму сознания,
  через феномены -> форма априорна, материя апостериорна. Например,
  думаешь, этот человек такой, а какой он на самом деле узнать
  невозможно. Это и есть вещь в себе.\\
  Категории:\\
  \begin{enumerate}
  \itemКоличества
    \begin{enumerate}
    \itemединство
    \itemмножество
    \itemсовокупность
    \end{enumerate}
  \itemКачества
    \begin{enumerate}
    \itemреальность 
    \itemотрицание 
    \itemограничение
    \end{enumerate}
  \itemОтношения
    \begin{enumerate}
    \itemсубстанция 
    \itemакциденция 
    \itemпричинодействие 
    \itemвзаимодействие
    \end{enumerate}
  \itemМодальность
    \begin{enumerate}
    \itemвозможность/невозможность 
    \itemсуществование/не существование 
    \itemнеобходимость/случайность
    \end{enumerate}
  \end{enumerate}
  Знания: априорные (время и пространство) и апостериорные.\\
  Суждения: аналитические(тавталогичные: все холостяки не женаты) и
  синтетические(получение новых выводов)\\
  Рассудок. Разум. Вера. \\
  Антиномии Канта(их 4):\\
  Есть Бог или нет.\\
  Мир делим до бесконечности или есть некие кванты.\\
  Все случайно или все не случайно.\\
  Кантовский переворот. В процессе познания нужно учитывать возможности
  человека. Вы пришли ловить рыбу. Забрасываете сеть с крупными ячейками
  и поймать крупное или ничего. А с мелкими ячейками вы могли бы поймать
  много мелкой рыбы. У всех разные сети, поэтому все зависит от
  возможностей.\\
  Категорический императив (этика) Канта. По большому счету поступай
  так, как хотел бы, чтобы поступали с тобой. Человек следует своим
  внутренним установкам всегда. Если у тебя есть принципы, то не надо
  придумывать себе оправданий или смотреть на других.\\
  ``Подходить к человеку не как к средству, а как к цели''.
\itemБентам, И. (1748-1832)\\
\itemФурье, Шарль (1772-1837)\\
  Социалист-утопист. Из многодетной и не слишком богатой семьи.\\
  Люди должны жить в фалангах. Труд не обременителен, все равны, мужчины
  равны женщинам, нет частной собственности. От каждого по способностям,
  каждому по потребностям.
\itemСен-Симон, Анри (1760-1825)\\
  Социалист-утопист. Активно участвовал в изменении жизни и не только своей
  страны(война за независимость США). Отделение церкви от
  государства. Трудиться должны все. Культ науки. Предлагал религию, где
  почитался Ньютон. Попытка классифицировать науки.
\itemОуэн, Роберт (1771-1858)\\
  Социалист-утопист. Был предпринимателем. Сократил рабочий день, ввел
  систему социальной поддержки (детсады, ясли), по вечерам проводились
  лекции, была библиотека. Если человек работает хорошо, то ему дают
  ``крутую'' карточку, по которой можно получить дополнительные ништяки
  с общего склада. Таким образом создал супер-производительное
  предприятие. Купил земли в США и создал социальные общины, но не
  сложилось - туда стекся всякий сброд и перессорился. Добился отмены
  труда детей.
\itemГегель, Георг Вильгельм Фридрих (1770-1831)\\
  Зажиточная семья. Получил образование. Протестантская семья. В
  университете учился на богословском факультете. Работал домашним
  учителем, возглавлял гимназию, преподавал в университете. Был
  рассеянным. Его никто не понимал сначала. Но со временем он начал
  говорить лучше и стал популярным лектором. Женился. Стал отцом троих
  детей. Был патриотом. Превозносил государство. Государство не Левиофан.\\
  Основное: абсолютная идея. Это некая сила, сопоставимая с Логосом,
  которая себя проявляет, согласно законам. Диалектика (у Гегеля) --
  учение о развитии. Мир развивается, он не статичен. У Канта познание
  ограничено тем, что мы приближаемся к знанию всего. Гегель критиковал
  Канта. Отделение феномена(сущности) от явления невозможно (все равно,
  что учиться плавать, не заходя в воду). Если мир непознаваем,
  то как же он развивается? \underline{Все действительное разумно, все
    разумное -- действительно.} Пытался все укладывать в триады.\\
  Наука логика. Тезис -> Антитезис -> Синтез. Идея в себе, она
  абстрактна, существует сама по себе.\\
  Бытие -- ничто. Потом становление идеи(проявление сущности). Формируется понятие. (Ребенок
  видит кактус -- нечто абстрактное. Укололся -- проявление
  сущности. Пришла мама и сказала: ``Это кактус'' -- возникло
  понятие).\\
  Идея проявляется в природе (\textbf{идея в ином}). Философия
  природы. Возьмем механику (идеи пространства и времени). Она проявляет
  себя в физических явлениях -- свет, тепло. Она проявляет себя в
  органике.\\
  Философия духа. Идея в себе и для себя. Философия субъективного
  (человек) -- антропология, феномен, психология. Философия объективного
  духа -- нравственность (семья, гражданское общество (не входит в
  государство, люди сами собрались и занимаются чем-либо (музыканты,
  пенсионеры против монетизации льгот, жители одного из домов против строительства парковки на
  месте детской площадки)), государство), право (по тому, какие законы,
  можно судить о государстве), этика. Философия абсолютного
  духа. Находит свое отражение в искусстве, религии и
  истории-философии.\\
  Диалектика(учение о развитии). 3 закона:
  \begin{enumerate}
  \itemЕдинство и борьба противоположностей (инь - янь, студент - преподаватель)
  \itemПереход количества в качество (обучение, количество времени
    кипения воды переходит в преобразование воды в пар)
  \itemОтрицание отрицания (одна мода отрицается другой модой, а та
    следующей и т.д.)
  \end{enumerate}
  Философия истории. Единство. Всеобщность. Степень свободы. От
  абстрактного идет к конкретному. Нация (как биологический
  организм). Юность, зрелость, умирание.\\
  Философия рациональна апеллирует к разуму, поэтому стоит выше
  религии. Религия апеллирует к чувствам. Естественная религия
  (например, язычество) дает больше свободы, поэтому она лучше
  христианства. Народная религия: индивидуальность и
  коллективизм. Христианство: авторитет и традиция. Христа уважал, но церковь считал международным
  политическим органом.\\
  С помощью категорий мы изучаем мир.\\
  Категории:
  \begin{enumerate}
  \itemОбщее и единое
  \itemПричина и следствие
  \itemВозмущение и действие
  \itemФорма и содержание
  \itemЦелое и часть
  \itemСущность и явление
  \itemНеобходимое и случайное
  \end{enumerate}
\itemЛессинг, Готхольд Эфраим (1729-1781)\\
Макс зачитал чего-то.
\itemГёте, Иоган Вольфган (так, нет?) (1749 - жив в наших сердцах)\\
Зажиточная семья, изучал языки, конную езду и фехтование. Читал много
старых книг из домашней бибилотеки и современную поэзию. Закончил
университет. Юрист. Движение ``Буря и натиск'' -- поэты. Встретился с
представителем движения. Поэзия должна идти от сердца. Поэт должен
писать о том, что пережил. Стал помощником какого-то крутого
чувака. Занимался наукой, геологией, минералогией. Потом уехал в
Италию, где приобщался к античной культуре, рисовал. Стал говорить,
что целью искусства является красота. Создал карту Германии, написал
метаморфозы растений, нашел какую-то кость в человеке. Убеждался в
родстве растений и животных. К концу жизни разочаровался воплотить в
жизнь античные идеалы и начал проповедовать реализм. Написал
``Фауста''.\\
Философом себя не считал. Не принимал отвлеченность. Встретился с
Шиллером. Они спорили. Шиллер базировался на Канте (идеализм чистого
разума), а Гёте описал метаморфозу растения. Может быть это и идея,
мне может быть приятно, что я имею идеи, не зная того, и даже вижу их
глазами. Все -- единое гармоническое целое. Развитие природы --
отбрасывание старого, отжившего свое. Эволюция. Человека можно считать
только одним из органов и нужно объединить все органы и ими
воспринимать -- это и будет восприятие Бога. Все фактическое уже есть
теория. Не нужно ничего искать за феноменами.
\\
Доп инфа у Равиля
\itemСерен Кьеркегор (1813~-~1855)\\
Истина -- не то, что ты знаешь, а то, что ты есть. Призывает
обратиться к человеку. Человек выше коллектива. Работа ``Или
или''. Человек перед Богом один на один. Не относится к
экзкстенционистам. Экзестенция -- существование человека,
судьба. Только в вере существование человека. Истинная вера -- как у
Авраама, который был готов пожертвовать сыном. Критикует Гегеля за то,
что его философия абстрактна и не дает ничего человеку. Человек не
должен становиться частью толпы. Три стадии становления человека:
\begin{enumerate}
\itemЭстетическая (чувстсвенная) -- удовольствия, получить от жизни все.
\itemЭтическая -- задумывается  о смысле жизни, испытывать
  ответственность за себя и других. Уповает на себя: считает, что все
  зависит от него. Если он будет делать все правильно, то все будет хорошо.
\itemРелигиозная -- понимает, что сам бывает бессилен в некоторых
  ситуациях. Иова. Иове Бог даровал все. Иова совершал много
  бескорыстных поступков, был праведником. Дьявол решил его проверить:
  отнял богатство, потом наслал проказу, потом отнял у него детей. Но
  Иова не отвернулся от Бога.
\end{enumerate}
Страх. Все мы чего-то боимся в большей или меньшей степени. Избавиться
от страха человек может только в вере.\\
Отношение к христианству. Прежде всего должно быть личное отношение к
Богу. Потом отношение к другим. И только потом культ. Люди играют в
христианство. То есть они делают все в обратном порядке.\\
``Театр. Собрались зрители. Пожар. Выходит клоун и говорит: ``Пожар!''. А
люди не верят и смеются.'' 
\itemШопенгауэр, Артур (1788~-~1860)\\
Немец. Детство сложное. Взаимоотношения в семье очень сложные. Большая
разница между родителями -- 20 лет. Отец был англофилом. Отец заставил
жену рожать Артура в Англии. Отец приумножал состояние, мать
развлекалась с друзьями. Артур чувствовал себя покинутым и
одиноким. Отец предложил Артуру: или ты отправляешься путешествовать с
нами и сдаешь экзамены экстерном, или заканчиваешь гимназию. Артур
выбрал первое. Отец показывал не только достопримечательности, но и
трущобы. Поездка оказала сильное влияние на мальчика. Отец стал
инвалидом, а мать продолжала устраивать салоны. Позже, отца
нашли в канаве. (Возможно, что это было самоубийство). Мать сказала: ``Вот у
тебя есть состояние отца. Можешь заниматься коммерческой
деятельностью, а можешь продолжать учиться.'' Он продолжил
учиться. Гёте часто посещал салоны матери Шопенгауэра. Он признавал,
что у нее есть литературный дар. А также имела дар создавать атмосферу
в компании. Не женился. Учился в университете. Защитил
диссертацию. Стал философом и начал преподавать. Шопенгауэр специально
поставил свои лекции в одно время с лекциями Гегеля. И остался без
студентов из-за этого.\\
Современная философия стала служанкой чужих интересов: государства,
коммерции. Против позитивизма. Уважал Платона и Канта. Открыл западу
восточную философию. ``Мир, как воля и представление''. Представление:
мы живем в некой системе, где мир представляется нам таким, каким мы
хотели бы или еще как-то. Мир -- это представление. Мир, как
воля. Миром управляет воля. Нет ничего разумного, управляющего. Воля
-- первоначало и абсолют. Но воля ни к чему не стремится, не имеет
цели, иррациональна, ее действия хаотичны. Пространство разделяет
людей, время убивает людей. Разум -- фикция. Мы развиваем науку, но не
ясно к чему это приведет. Мы живем, как во сне. Мир нам как-то
представляется. Куда денется ваш оптимизм, когда вы пройдетесь по
психбольницам, больницам, где умирают, тюрьмам? Благодаря телу мы
испытываем страдание и наслаждение. Мы стремимся уменьшить страдание,
увеличить наслаждение. Это нормально. Человек это маятник. Он либо
страдает, либо страдает от того, что нечем себя занять. Жизнь --
непрерывная борьба за существование. Человек -- животное дикое и
жестокое. Животные не способны испытывать удовольствия наслаждаясь
видом чужих страданий, а люди могут. А что делать, раз все так плохо?
Есть два пути: аскеза и искусство. При аскезе нечего терять, а с
помощью искусства можно уйти в другой мир. Отдает предпочтение
интуиции перед разумом. Человек стремится к благу. Блага есть внешние
и телесные, духовные. Государство служит для
укрощения хищного человека и превращения его в травоядное.\\
Мир управляется волей. Воля – первоначало и абсолют, она управляет
миром, она иррациональна и нелогична. Разум – фикция. Жизнь –
постоянная борьба с существованием (возможно существование здесь
аналог скуки, а может и именно существование). Был пессимист
(официальный термин) считал, что прогресс – дерьмо собачье. Есть 2
выхода из этого ужаса – разумная аскеза, самоограничение, и искусство,
уход от действительности в искусство. Интуиция же, с другой стороны –
солнце, согревающее мир. 3 вида блага – внешние (богатство), телесные
(здоровье, красота, духовные) и духовные. Благо здесь не обязательно
означает хорошо или полезно.
\item Ницше, Фридрих (1844~-~1900).\\
Идея сверхчеловека. Написал на эту тему книжку, которая никому не понравилась. В человеке есть 2 начала: Аполоновское ( сила, разум, порядок), Дионисское  (инстинктивное, иррациональное) Вот второе и есть сверхчеловечекое. Лозунг – Бог умер. Активный нигилизм – знак повышенной мощи духа. Но хуже всего – деятельное сострадание.
\itemМаркс, Карл (1818~-~1883).\\
Где-то в его времена было такое разделение по политическим идеям:
\begin{itemize}
\itemСоциал-демократы
\begin{itemize}
\itemБольшая роль государства
\itemАтеизм
\end{itemize}
\itemЛиберализм
\begin{itemize}
\itemЧастная собственность
\itemГосударство лишь исполняет роль ночного сторожа
\itemСвободная религия
\itemИзменения эволюционны
\end{itemize}
\itemКонсерватизм, в основном поддерживаемый аристократами
\begin{itemize}
\itemЧеткая иерархия общества
\itemСтабильность
\itemЧастная собственность
\itemЕдиная религия
\end{itemize}
\end{itemize}
Со временем Либералы поняли, что государство все-таки должны вмешиваться, Соц.Демократы пересмотрели тезис диктатуры пролетариата. Консерваторы позиций не сдали:)\\
Социал-демократы: Отказ от частной собственности, распределение богатства между всеми в
равной мере, национализация имущества. Центральная роль государства,
религия отрицается. Революционный метод развития.
Либерализм: государству отводится роль ночного сторожа (осуществление
прав), оно не вмешивается в жизнь людей, частная собственность,
муниципальная собственность, свобода вероисповедания, переустройство
общества реформами (эволюционно).
Консервативная линия: аристократы, чиновники. За четкую иерархию,
изменения незначительные, а лучше без них. За религию, которая
поддерживает государственную власть. Частная собственность.
\\
Либералы поняли, что государство должно вмешиваться в
экономику. Консерваторы -- сохранить все, что есть, поняли, что все
течет, все меняется. Социал-демократы пересмотрели тезис о диктатуре
пролетариата.
\\
\item Конт, Огюст (1798~-~1857)\\
В 9 отдали в лицей и вступал в конфликты постоянно и это привело к
тому, что лицей закрыли. В 15 поступил в политехническую школу. Стал
преподавать, но карьеру не сделал. Начал преподавать позитивную
философию. По которой он прочитал две лекции. Позже начал писать труды
и рассылать их европейским монархам. Никто не оценил из них. Со
временем у него появились поклонники, которые его финансировали. И он
это воспринимал, как обязанность. Если финансирование заканчивалось,
он требовал его возобновить.\\
Нужно отказаться от метафизики. Философия должна заниматься только
теми дисциплинами и фактами, которые известны науке в данный
момент. Философия -- это методология. Считается отцом социологии. Не
только можно, но и нужно применять естественно-научные методы ко
всякой хрени. Отношения в обществе и как они развиваются. Социальная
статика и динамика. Классификация наук: математика, астрономия,
физика, химия, биология, социология (социальная физика) (От низшего к
высшему). Этот ряд составляет позитивную науку. Теологическая стадия
(конкретный уровень -- есть статуя, ей поклоняются). Метафизический
уровень (Бог приобретает абстрактные черты). Истинный уровень
(позитивная наука). Религию нужно отринуть. Культ науки. Каждый месяц
именовать позитивными богами или личностями (Прометей,
например). Культ факта. Было священное кресло, где сидела его жена
-- алтарь. \\
Был очень нервный. Считал что:
\begin{itemize}
\itemНужно отказаться от метафизики и полагаться лишь на четкие научные факты
\itemФилософия сводится к методология
\end{itemize}
Конт является отцом социологии, придумал социальную статистику, социальную.динамику, ввел социальные институты.
Выделял основные науки: Математика, Астрономия, Физика, Химия, Биология, (конечно) Социология.
Выделял три стадии сознания человечества: 1-я –  теологическая, во второй, метафизической, Бог – уже более абстрактная штука, скорее, чем реальная и 3-я – научная, она научная. 
Религия – культ науки.
\itemМилль, Джон Стюард (1806~-~1873)\\
Воспитывался отцом, был вундеркиндом. Боролся за избирательные
права. Предпочтение отдавал индукции. Этическая проблема (добро и зло)
должна рассматриваться в историческом контексте.
\itemСпенсер, Герберт (1820~-~1903)\\
Родился в семье потомственных педагогов. Интереса к учебе не
проявлял. Зато занимался самообразованием и достиг успехов. Был
инженером. \\
Предлагал отказаться от теологических объяснений исторических
событий. Дарвин считал его отцом идеи эволюции. Общество -- организм,
части которого взаимосвязаны. Общество самовосстановится.\\
Первым заговорил об эволюции, об обществе, как о едином организме.\\
\textbf{2-я волна позитивизма:}
\itemГёрдер, Иоганн-Ботеррку(?) (1744~-~1803)\\
Учился у Канта. Знаком с Ламбером. Трактат о происхождении
языков. Получил премию Берлинского университета и познакомился с
Гёте. Стал первым проповедником где-то там. Философия и история
человечества.
\\Больше информации у Кирилла Рудакова\\
Второй позитивизм
\itemМах, Эрнст(1838~-~1916)\\
В 26 лет уже был профессором в Праге. Закончил Венский Университет.\\
Первые работы посвящены слуху и зрению. Позже сверхзвук. Методология:
учение о нейтральных элементах. Все физическое я могу разложить на
элементы неразложимые. Нейтральные элементы -- комплексы
ощущений. Задача науки: т.к. она описывает ощущения, то нужно
отказаться от поиска причины, субстанции. Цель науки: приспособить
теорию к действительности. Критерий истинности заменяется критерием
успешности. Принцип экономии мышления. Бритва Оккама. Этот принцип
связан со стремлением человека с самосохранению. Нужно избирать самый
легкий путь решения проблемы. Критикует ньютоновскую механику,
критикует теорию Эйнштейна, не верит в существование атома.\\
Сделал  учение о нейтральных элементах, нейтральный элемент – комплекс ощущений.
Говорил, что следует отказаться от понятий причинности и субстанции, как тупости несусветной, так как наука только описывает, а не выясняет зачем, да что было первее. 
\itemДю Гем, Пьер (1861~-~1916)\\
Наука описания. Теория -- средство экономии мышления. Физическая
теория не есть объяснение. Научный метод:
\begin{enumerate}
\itemЭкспериментальные факты. 
\itemЭкспериментальные законы.
\itemСоздание теории.
\end{enumerate}
Или в другую сторону:
\begin{itemize}
\itemОпределить физические величины
\itemВыбрать теорию
\itemВыбрать или создать матаппарат
\itemСравнить с опытом
\end{itemize}
Подчеркивает роль интуиции. Определение и измерение физических
величин, выбор гипотез, математическое развитие теории, сравнение
теории с опытом. Недозагруженность теории.
\itemПуанкаре, Анри (1864~-~1912)\\
Конвенционализм. Приобретение знаний -- задача экспериментальной
физики, а составление каталога (систематизация) -- математической
физики.
Истина -- это то, что ближе всего к истине (нам так кажется).\\
Принцип конвенционализма: ученые лишь договорились о критериях истинности. Экспериментальная физика добывает факты. Мат физика – составление каталога.
\end{itemize}
\end{document}
